\subsection{Mass Functions}

\textbf{key}: \texttt{/mass\_function}

\subsubsection{Datasets}

\begin{center}
\begin{table}[H]
\begin{tabular}{ | c | c | c | c | }
    \hline
    Variable & Dataset Name & Supplementary Datasets &  attributes \\
    \hline\hline
    Number of stars & \texttt{N\(^*\)} & \texttt{\(\Delta\)N\(^*\)} & \\
    \hline
    Radial bin inner bound & \texttt{r1\(^*\)} & & \texttt{unit} \\
    \hline
    Radial bin outer bound & \texttt{r2\(^*\)} & & \texttt{unit} \\
    \hline
    Mass bin inner bound & \texttt{m1\(^*\)} & & \texttt{unit} \\
    \hline
    Mass bin outer bound & \texttt{m2\(^*\)} & & \texttt{unit} \\
    \hline
    Observation fields & \texttt{fields\(^*\)} & &
    \texttt{field\_unit}\\ &&& See caption \\
    \hline
\end{tabular}
\caption*{
    The \texttt{fields} dataset is an array of string names of different
    observational program's PIs. Each name has a corresponding attribute entry
    within the dataset, which consists of a 2d-array of (RA, DEC) coordinates
    which define the polygonal boundaries of this observation. If a single
    program has multiple polygons, they are denoted by the addition of an
    underscore followed by a single alphanumeric character (\_a, \_b, etc.).
    All coordinates are given in the units defined by \texttt{field\_unit}.
}
\end{table}
\end{center}



\subsubsection{Attributes}

\begin{center}
\begin{table}[H]
\begin{tabular}{ | c | c | }
    \hline
    Attribute & Description \\
    \hline\hline
    \texttt{source} & Literature source(s) of data \\
    % TODO each field technically has it's own source as well
    \hline
\end{tabular}
\end{table}
\end{center}
