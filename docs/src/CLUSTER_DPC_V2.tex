\documentclass[12pt]{article}

\usepackage[
    a4paper, headsep=1.5cm, headheight=30pt,
    left=2.5cm,right=2.5cm,top=4cm,bottom=3cm]{geometry}
\usepackage{fancyhdr}
\usepackage{enumitem}

\usepackage{amsmath}
\usepackage{siunitx}
\usepackage{graphicx}
\usepackage[font=footnotesize]{caption}
\usepackage{float}
\graphicspath{ {./figures/} }


\begin{document}

\begin{titlepage}
       \vspace*{2cm}

       \LARGE
        GCfit

       \vspace{2cm}

       \Huge
       \textbf{Globular Cluster Observation Data}

       \vspace{2cm}
        
        \LARGE
        Data File Catalog

       \vspace{1.5cm}

       \vfill

       Version 1
\end{titlepage}

\section{Introduction}

in a hdf file....... etc

All supplementary error datasets can be either the symmetric dataset or two
seperate down and up error datasets.

Everything should be within the given "key" group under the main file group.
But if multiple "versions" of the datasets are to be used, then you can put
everything under other groups under the key, which should be sorted out
correctly under data.

But, this MUST be done for everything in that group, all parent groups of
subgroups will not be read in data, so there can be no shared space for groups
and datasets. All datasets must go under the lowest level of subgroup.

\section{Attributes}

Overall cluster attributes. 

\begin{center}
\begin{table}[H]
\begin{tabular}{ | c | c | c | c | c | }
    \hline
    Variable & Attribute Name & Notes & Default Value & Units \\
    \hline\hline
    Galactic Longitude & \texttt{l} &
    Required for pulsar fitting & N/A & degrees \\
    \hline
    Galactic Latitude & \texttt{b}  &
    Required for pulsar fitting & N/A & degrees \\
    \hline
    Right Ascension & \texttt{RA} &
    Required for mass function fitting & N/A & degrees \\
    \hline
    Declination & \texttt{DEC}  &
    Required for mass function fitting & N/A & degrees \\
    \hline
    Metallicity & \texttt{FeHe} &
    Defines mass function evolution & -1.00 & dex \\
    \hline
    Age & \texttt{age} & Defines mass function evolution & 12 & Gyr \\
    \hline
    Total Proper Motion & \texttt{\(\mu\)} &
    Required for pulsar fitting & N/A & mas/yr \\
    \hline
    Total escape rate \(\dot{N}\) & \texttt{Ndot} &
    Defines mass function evolution & -20 & \\
    \hline
\end{tabular}
\end{table}
\end{center}
% TODO i dont know if initials should really go here, I need to but cluster
%   stuff like Z and coords and overall pm in here (theyre actually attrs)
% TODO document the sources of ^ somehow (and units as well)


\textbf{key}: \texttt{/initials}

All the parameters which are fit on

these values are the initial guesses

defaults are used if this isnt in the file, or any field is missing


\begin{center}
\begin{table}[H]
\begin{tabular}{ | c | c | c | c | }
    \hline
    Variable & Attribute Name & Description & Default Value \\
    \hline\hline
    \(W_0\) & \texttt{W0} & Central potential & 6.0 \\
    \hline
    M & \texttt{M} & Total cluster mass [\(10^6 M_{\odot}\)] & 0.69 \\
    \hline
    \(r_h\) & \texttt{rh} & Half-mass radius [pc] & 2.88 \\
    \hline
    \(\log(r_a)\) & \texttt{ra} & Anisotropy radius [\(\log(pc)\)] & 1.23 \\
    \hline
    g & \texttt{g} & Truncation parameter & 0.75 \\
    \hline
    \(\delta\) & \texttt{delta} & & 0.45 \\
    \hline
    \(s^2\) & \texttt{s2} & Velocity scale nuisance parameter & 0.1 \\
    \hline
    F & \texttt{F} & Mass function nuisance parameter & 0.45 \\
    \hline
    \(a_1\) & \texttt{a1} & 1st mass function power law exponent & 0.5 \\
    \hline
    \(a_2\) & \texttt{a2} & 2nd mass function power law exponent & 1.3 \\
    \hline
    \(a_3\) & \texttt{a3} & 3rd mass function power law exponent & 2.5 \\
    \hline
    \(BH_{ret}\)&\texttt{BHret} & Black hole initial retention fraction & 0.5 \\
    \hline
    d & \texttt{d} & Cluster distance [kpc] & 6.405 \\
    \hline
\end{tabular}
\end{table}
\end{center}

\section{Data Products}

* denotes required fields

\subsection{Pulsar Accelerations}

\textbf{key}: \texttt{/pulsar}

\subsubsection{Datasets}

\begin{center}
\begin{tabular}{ | c | c | c | c | }
    \hline
    Variable & Dataset Name & Supplementary Datasets &  attributes \\
    \hline\hline
    Radial distance & \texttt{r\(^*\)} & & \texttt{units} \\
    \hline
    LOS acceleration & \texttt{a\_los\(^*\)} & \texttt{\(\Delta\)a\_los} &
    \texttt{units}\\ &&&\texttt{method}\\ &&&\texttt{a\_g}\\ &&&\texttt{a\_s}\\
    \hline
    Intrinsic acceleration & \texttt{a\_int} & & 
    \texttt{units} \\ & & & \texttt{method} \\ & & & \texttt{B} \\
    \hline
    Spin period & \texttt{P} & \texttt{\(\Delta\)P} & \texttt{units} \\
    \hline
    Spin period derivative & \texttt{dP\_meas} & \texttt{\(\Delta\)dP\_meas} &
    \texttt{units} \\
    \hline
    Pulsar identifier & \texttt{id} & & \\
    \hline
\end{tabular}
\end{center}

\subsubsection{Attributes}


\begin{center}
\begin{tabular}{ | c | c | }
    \hline
    Attribute & Description \\
    \hline\hline
    \texttt{source} & Literature source(s) of data \\
    \hline
\end{tabular}
\end{center}


\subsection{Number Density}

\textbf{key}: \texttt{/number\_density}

\subsubsection{Datasets}

\begin{center}
\begin{tabular}{ | c | c | c | c | }
    \hline
    Variable & Dataset Name & Supplementary Datasets &  attributes \\
    \hline\hline
    Radial distance & \texttt{r} & & \texttt{units} \\
    \hline
    Number Density & \texttt{\(\Sigma\)} & \texttt{\(\Delta\Sigma\)} & \texttt{units} \\
    \hline

\end{tabular}
\end{center}

\subsubsection{Attributes}


\begin{center}
\begin{tabular}{ | c | c | }
    \hline
    Attribute & Description \\
    \hline\hline
    \texttt{source} & Literature source(s) of data \\
    \hline
\end{tabular}
\end{center}


\subsection{Proper Motions}

\textbf{key}: \texttt{/proper\_motion}

\subsubsection{Datasets}

\begin{center}
\begin{table}[H]
\begin{tabular}{ | c | c | c | c | }
    \hline
    Variable & Dataset Name & Supplementary Datasets &  attributes \\
    \hline\hline
    Radial distance & \texttt{r\(^*\)} & \texttt{\(\Delta\)r} & \texttt{unit} \\
    \hline
    Total proper motion & \texttt{PM\_tot\(^*\)} &
    \texttt{\(\Delta\)PM\_tot\(^*\)} & \texttt{unit}\\
    \hline
    Proper motion ratio & \texttt{PM\_ratio\(^*\)} &
    \texttt{\(\Delta\)PM\_ratio\(^*\)} & \texttt{method} \\
    \hline
    Radial proper motion & \texttt{PM\_R\(^*\)} &
    \texttt{\(\Delta\)PM\_R\(^*\)} & \texttt{unit}\\
    \hline
    Tangential proper motion & \texttt{PM\_T\(^*\)} &
    \texttt{\(\Delta\)PM\_T\(^*\)} & \texttt{unit}\\
    \hline
\end{tabular}
\caption*{
    The proper motions can be fit on any of these components, alone or as a
    group. The corresponding errors are required for any.
}
\end{table}
\end{center}

\subsubsection{Attributes}


\begin{center}
\begin{table}[H]
\begin{tabular}{ | c | c | }
    \hline
    Attribute & Description \\
    \hline\hline
    \texttt{source} & Literature source(s) of data \\
    \hline
    \texttt{m} & Mean stellar mass of tracer stars [\(M_\odot\)] \\
    \hline
\end{tabular}
\end{table}
\end{center}


\subsection{Velocity Dispersions}

\textbf{key}: \texttt{/velocity\_dispersion}

\subsubsection{Datasets}

\begin{center}
\begin{tabular}{ | c | c | c | c | }
    \hline
    Variable & Dataset Name & Supplementary Datasets &  attributes \\
    \hline\hline
    Radial distance & \texttt{r\(^*\)} & & \texttt{units} \\
    \hline
    LOS velocity dispersion & \texttt{\(\sigma^*\)} & \texttt{\(\Delta\sigma^*\)} &
    \texttt{units}\\
    \hline
\end{tabular}
\end{center}

\subsubsection{Attributes}


\begin{center}
\begin{tabular}{ | c | c | }
    \hline
    Attribute & Description \\
    \hline\hline
    \texttt{source} & Literature source(s) of data \\
    \hline
    \texttt{m} & Mean stellar mass of tracer stars [\(M_\odot\)] \\
    \hline
\end{tabular}
\end{center}


\subsection{Mass Functions}

\textbf{key}: \texttt{/mass\_function}

\subsubsection{Datasets}

\begin{center}
\begin{table}[H]
\begin{tabular}{ | c | c | c | c | }
    \hline
    Variable & Dataset Name & Supplementary Datasets &  attributes \\
    \hline\hline
    Number of stars & \texttt{N\(^*\)} & \texttt{\(\Delta\)N\(^*\)} & \\
    \hline
    Radial bin inner bound & \texttt{r1\(^*\)} & & \texttt{unit} \\
    \hline
    Radial bin outer bound & \texttt{r2\(^*\)} & & \texttt{unit} \\
    \hline
    Mass bin inner bound & \texttt{m1\(^*\)} & & \texttt{unit} \\
    \hline
    Mass bin outer bound & \texttt{m2\(^*\)} & & \texttt{unit} \\
    \hline
    Observation fields & \texttt{fields\(^*\)} & & See caption \\
    \hline
\end{tabular}
\caption*{
    The \texttt{fields} dataset is an empty dataset and exists only as a
    container for it's attributes. Each mass function will have some number of
    field boundary polygon attributes, each denoted by a single alphanumeric
    character (a, b, c, etc.). Each attribute consists of a 2d-array of
    (RA, DEC) coordinates which define the polygonal boundaries of this
    observation
}
\end{table}
\end{center}



\subsubsection{Attributes}

\begin{center}
\begin{table}[H]
\begin{tabular}{ | c | c | }
    \hline
    Attribute & Description \\
    \hline\hline
    \texttt{source} & Literature source(s) of data \\
    \hline
    % TODO units should be stored/documented in the field, or here, not both
    \texttt{field\_unit} & Coordinate units of all \texttt{fields} boundaries \\
    \hline
\end{tabular}
\end{table}
\end{center}



\end{document}

