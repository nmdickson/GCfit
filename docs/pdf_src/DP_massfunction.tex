\subsection{Mass Functions}

\textbf{key}: \texttt{/mass\_function}

\subsubsection{Datasets}

\begin{center}
\begin{table}[H]
\begin{tabular}{ | c | c | c | c | }
    \hline
    Variable & Dataset Name & Supplementary Datasets &  attributes \\
    \hline\hline
    Number of stars & \texttt{N\(^*\)} & \texttt{\(\Delta\)N\(^*\)} & \\
    \hline
    Radial bin inner bound & \texttt{r1\(^*\)} & & \texttt{unit} \\
    \hline
    Radial bin outer bound & \texttt{r2\(^*\)} & & \texttt{unit} \\
    \hline
    Mass bin inner bound & \texttt{m1\(^*\)} & & \texttt{unit} \\
    \hline
    Mass bin outer bound & \texttt{m2\(^*\)} & & \texttt{unit} \\
    \hline
    Observation fields & \texttt{fields\(^*\)} & & See caption \\
    \hline
\end{tabular}
\caption*{
    The \texttt{fields} dataset is an empty dataset and exists only as a
    container for it's attributes. Each mass function will have some number of
    field boundary polygon attributes, each denoted by a single alphanumeric
    character (a, b, c, etc.). Each attribute consists of a 2d-array of
    (RA, DEC) coordinates which define the polygonal boundaries of this
    observation
}
\end{table}
\end{center}



\subsubsection{Attributes}

\begin{center}
\begin{table}[H]
\begin{tabular}{ | c | c | }
    \hline
    Attribute & Description \\
    \hline\hline
    % TODO each field technically has it's own source as well
    \texttt{source} & Literature source(s) of data \\
    \hline
    \texttt{field\_unit} & Coordinate units of all \texttt{fields} boundaries \\
    \hline
\end{tabular}
\end{table}
\end{center}
