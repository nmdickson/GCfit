\subsection{Pulsar Accelerations}

\textbf{key}: \texttt{/pulsar}

\subsubsection{Datasets}

\begin{center}
\begin{table}[H]
\begin{tabular}{ | c | c | c | c | }
    \hline
    Variable & Dataset Name & Supplementary Datasets &  attributes \\
    \hline\hline
    Radial distance & \texttt{r\(^*\)} & & \texttt{unit} \\
    \hline
    Spin period & \texttt{P\(^*\)} & \texttt{\(\Delta\)P} & \texttt{unit}\\
    \hline
    Spin period derivative & \texttt{Pdot\_meas\(^*\)} &
    \texttt{\(\Delta\)Pdot\_meas\(^*\)} & \texttt{unit} \\
    \hline
    Orbital period & \texttt{Pb\(^*\)} & \texttt{\(\Delta\)Pb} & \texttt{unit}\\
    \hline
    Orbital period derivative & \texttt{Pbdot\_meas\(^*\)} &
    \texttt{\(\Delta\)Pbdot\_meas\(^*\)} & \texttt{unit} \\
    \hline
    Dispersion Measure & \texttt{DM} & \texttt{\(\Delta\)DM} & \texttt{unit} \\
    \hline
    Pulsar identifier & \texttt{id} & & \\
    \hline
\end{tabular}
\caption*{
    Pulsars can be fit on the timing solutions of both the isolated
    pulsar spin (P, Pdot\_meas) and the binary systems orbit (Pb, Pbdot).
    The period, derivative and corresponding errors are required for either.
    Optionally, the dispersion measure (DM) can be used to provide a better
    acceleration constraint, where available.
}
\end{table}
\end{center}

\subsubsection{Attributes}


\begin{center}
\begin{table}[H]
\begin{tabular}{ | c | c | }
    \hline
    Attribute & Description \\
    \hline\hline
    \texttt{source} & Literature source(s) of data \\
    \hline
    \texttt{m} & Mean stellar mass of tracer stars [\(M_\odot\)] \\
    \hline
\end{tabular}
\end{table}
\end{center}
